\chapter{Bilan}

\section{Écarts avec le cahier des charges}

Le projet a globalement su respecter les contraintes posées par le cahier des charges. L'ensemble des fonctions principales a pu être intégré au projet, et la majorité des fonctionnalités optionnelles le sont aussi. Seules les prédictions concernant la charge de travail sont largement fausses.

\subsection{Fonctionnalités}
Notre cahier des charges contenait huit fonctionnalités distinctes : entraîner l'agent, tester l'agent, regarder l'agent jouer, sauvegarder l'agent, charger l'agent, sélectionner l'environnement,  sélectionner un type d'algorithme d'apprentissage par renforcement, et jouer manuellement à l'environnement.
\par
Les sept premières sont totalement implémentées et fonctionnent sans problème. Seule la dernière, la possibilité de faire jouer un humain, n'a pas pu être ajoutée par contrainte temporelle. Toutefois sa priorité était de ``Would'', soit la priorité la plus basse, cela ne pose donc pas de soucis particulier (la fonctionnalité étant mentionnée comme ``bonus'' puisque n'ayant aucun rapport direct avec l'apprentissage par renforcement).
\par
Ces fonctionnalités sont présentes dans la maquette, et le projet actuel respecte globalement assez fidèlement le déroulement de la maquette initiale.


\subsection{Planification}
Cette partie du cahier des charges n'a en revanche absolument pas réussi à structurer le déroulement réel du projet, qui s'est en conséquence montré beaucoup plus chaotique.
\par
Comme mentionné dans la partie dédiée, la planification ne correspondait absolument pas à la réalité. La majeure partie du temps fut occupée par les recherches sur les algorithmes ainsi que sur les essais d'entraînements. Ces derniers sont particulièrement chronophages et l'entraînement d'un agent DQN sur un jeu Atari nécessite sur nos machines au moins 2h pour voir de vrais résultats, et environ le double pour espérer atteindre son niveau d'amélioration maximal.
\par 
L'organisation s'en est retrouvée complètement chamboulée, il nous était impossible de suivre les taches imposées dans l'ordre imposé et selon les durées estimées.


\section{Gestion de projet}
La gestion du projet, son cahier des charges, les analyses, et globalement l'ensemble des phases constituant le cycle de vie d'un projet sont des notions que nous avions déjà abordée dans une UE du semestre dernier. Toutefois nous étions dans celle-ci bien plus guidée, et le projet n'était pas libre.
\par
Nous avons donc dû dans ce projet effectuer la majorité des décisions en autonomie (y compris définir les besoins). En conséquence, notre organisation était bien moins cohérente et solide cette fois ci. Mais nous avons aussi pu mieux expérimenter les enjeux réels d'un projet, et comprendre l'importance des différentes phases d'organisation.

\section{Connaissances}
Bien que cela ne se soit pas passé dans la rigueur du cadre universitaire, ce projet nous a permis d'acquérir un nombre conséquent de connaissances. Il nous était impensable au début de ce projet que nous puissions écrire ``Principe des algorithmes utilisés'' actuellement présente dans le rapport.
\par
Nos expérimentations et les difficultés que nous avons rencontré nous ont forcé à chercher des réponses et des connaissances, en particulier lors de la partie sur l'application du DQN à Pac-Man. Cette période nous a demandé de revoir tous nos pré-acquis et de questionner nos choix sur chaque élément constituant l'agent.
\par
Ce projet nous a permis en définitive d'obtenir des bases assez larges en apprentissage par renforcement ainsi qu'en apprentissage profond.
